\documentclass[addpoints]{exam}
\usepackage{preamble}
\sisetup{group-separator = {,}}
\setlength\parskip{5pt}

\pagestyle{headandfoot}
\runningheadrule


\firstpagefooter{Access for free at \href{https://openstax.org/books/astronomy-2e/pages/1-introduction}{https://openstax.org/books/astronomy-2e/pages/1-introduction}}{}{}
\runningfooter{Access for free at \href{https://openstax.org/books/astronomy-2e/pages/1-introduction}{https://openstax.org/books/astronomy-2e/pages/1-introduction}}{}{}


\firstpageheader{Astronomy}{Problem Set}{Chapter 5: {\small Radiation and Spectra}}


\CorrectChoiceEmphasis{\color{red}\bfseries}
\SolutionEmphasis{\color{red}}
\printanswers

\begin{document}
\begin{questions}

\question
What distinguishes one type of electromagnetic radiation from another?

\question
Explain why light is referred to as electromagnetic radiation.

\question
Which type of wave has a longer wavelength: AM radio waves (with frequencies in the kilohertz range) or FM radio waves (with frequencies in the megahertz range)? Explain.

% \question
% Explain why astronomers long ago believed that space must be filled with some kind of substance (the ``aether'') instead of the vacuum we know it is today.

\question
Which is more dangerous to living things: gamma rays or X-rays? Explain.

\question
Why is it dangerous to be exposed to X-rays but not (or at least much less) dangerous to be exposed to radio waves?

% \question
% Explain why we have to observe stars and other astronomical objects from above Earth's atmosphere in order to fully learn about their properties.

\question
Explain why hotter objects tend to radiate more energetic photons compared to cooler objects.

\question
Explain how we can deduce the temperature of a star by determining its color.

\question
Explain what Joseph Fraunhofer discovered about stellar spectra.

\question
With what type of electromagnetic radiation would you observe a star with a temperature of \SI{5800}{K}?

\question
With what type of electromagnetic radiation would you observe a gas heated to a temperature of one million K?

\question
With what type of electromagnetic radiation would you observe a person on a dark night?


\question
Go outside on a clear night, wait 15 minutes for your eyes to adjust to the dark, and look carefully at the brightest stars. Some should look slightly red and others slightly blue. The primary factor that determines the color of a star is its temperature. Which is hotter: a blue star or a red one? Explain

\question
Water faucets are often labeled with a red dot for hot water and a blue dot for cold. Given Wien's law, does this labeling make sense?

\question
What is the wavelength of the carrier wave of a campus radio station, broadcasting at a frequency of \SI{97.2}{MHz} (million cycles per second or million hertz)?

\question
What is the frequency of a red laser beam, with a wavelength of \SI{670}{nm}, which your astronomy instructor might use to point to slides during a lecture on galaxies?

\question \label{ques:blacklight}
You go to a dance club to forget how hard your astronomy midterm was. What is the frequency of a wave of ultraviolet light coming from a blacklight in the club, if its wavelength is \SI{150}{nm}?

\question
What is the energy of the photon with the frequency you calculated in question~\ref{ques:blacklight}?

\question
If the emitted infrared radiation from Pluto has a wavelength of maximum intensity at \SI{75000}{nm}, what is the temperature of Pluto assuming it follows Wien's law?

\question
What is the temperature of a star whose maximum light is emitted at a wavelength of \SI{290}{nm}?



\end{questions}
\end{document}