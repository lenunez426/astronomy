\documentclass{article}
\usepackage{preamble}
\setlength\parskip{5pt}
\DeclareSIUnit\angstrom{\text {Å}}

\makenoidxglossaries

\newglossaryentry{absorption spectrum}{
    name=absorption spectrum,
    description={a series or pattern of dark lines superimposed on a continuous spectrum}
}

\newglossaryentry{blackbody}{
    name=blackbody,
    description={an idealized object that absorbs all electromagnetic energy that falls onto it}
}

\newglossaryentry{emission spectrum}{
    name=emission spectrum,
    description={a series or pattern of bright lines superimposed on a continuous spectrum}
}

\newglossaryentry{photon}{
    name=photon,
    description={a discrete unit (or ``packet'') of electromagnetic energy}
}

\newglossaryentry{visible light}{
    name=visible light,
    description={electromagnetic radiation with wavelengths of roughly 400--700 nanometers; visible to the human eye}
}

\newglossaryentry{wavelength}{
    name=wavelength,
    description={the distance from crest to crest or trough to trough in a wave}
}

\newglossaryentry{electromagnetic radiation}{
    name=electromagnetic radiation,
    description={radiation consisting of waves propagated through regularly varying electric and magnetic fields and traveling at the speed of light}
}

\newglossaryentry{frequency}{
    name=frequency,
    description={the number of waves that cross a given point per unit time (in radiation)}
}

\newglossaryentry{electromagnetic spectrum}{
    name=electromagnetic spectrum,
    description={the whole array or family of electromagnetic waves, from radio to gamma rays}
}

\newglossaryentry{gamma rays}{
    name=gamma rays,
    description={photons (of electromagnetic radiation) of energy with wavelengths no longer than 0.01 nanometer; the most energetic form of electromagnetic radiation}
}

\newglossaryentry{X-rays}{
    name=X-rays,
    description={electromagnetic radiation with wavelengths between 0.01 nanometer and 20 nanometers; intermediate between those of ultraviolet radiation and gamma rays}
}

\newglossaryentry{ultraviolet}{
    name=ultraviolet,
    description={electromagnetic radiation of wavelengths 10 to 400 nanometers; shorter than the shortest visible wavelengths}
}

\newglossaryentry{infrared}{
    name=infrared,
    description={electromagnetic radiation of wavelength $10^3$ -- $10^6$ nanometers; longer than the longest (red) wavelengths that can be perceived by the eye, but shorter than radio wavelengths}
}

\newglossaryentry{microwave}{
    name=microwave,
    description={electromagnetic radiation of wavelengths from 1 millimeter to 1 meter; longer than infrared but shorter than radio waves}
}

\newglossaryentry{radio waves}{
    name=radio waves,
    description={all electromagnetic waves longer than microwaves, including radar waves and AM radio waves}
}

\newglossaryentry{Wien's law}{
    name=Wien's law,
    description={formula that relates the temperature of a blackbody to the wavelength at which it emits the greatest intensity of radiation}
}

\newglossaryentry{dispersion}{
    name=dispersion,
    description={separation of different wavelengths of white light through refraction of different amounts}
}

\newglossaryentry{spectrometer}{
    name=spectrometer,
    description={an instrument for obtaining a spectrum; in astronomy, usually attached to a telescope to record the spectrum of a star, galaxy, or other astronomical object}
}

\newglossaryentry{continuous spectrum}{
    name=continuous spectrum,
    description={a spectrum of light composed of radiation of a continuous range of wavelengths or colors, rather than only certain discrete wavelengths}
}

\newglossaryentry{wave}
{
    name=wave,
    description=a disturbance that moves from a source and carries energy
}

\newglossaryentry{wave cycle}{
    name = wave cycle,
    description= the potion of a wave encompassed by 1 wavelength
}


\begin{document}

Astronomy \hfill Lecture Notes \hfill Chapter 5: Radiation and Spectra

\section*{Spectroscopy in Astronomy}

1672: Isaac Newton splits sunlight (white light) into colors of rainbow (``spectrum'') using prism


spectrometer: instrument that produces a spectrum from incoming light

1802: William Wollaston makes improved spectrometer

Uses lens to project solar spectrum on large screen {\color{lightgray} (What do I mean by ``solar''?)}

Sees dark lines and ranges of missing color 

1815: Joseph Fraunhofer counts 600 dark lines (``Fraunhofer lines'') in solar spectrum

Why dark lines? Atoms/molecules in Sun interact with electromagnetic radiation (light)

Different elements (e.g., helium, calcium) absorb different wavelengths of color

Fraunhofer lines are spectral signatures (fingerprints) of atoms

% spectroscopy: the measurement and analysis of a spectrum 

% Spectroscopy tells us what Sun, other stars are made of

\framebox{\textbf{Practice}} \textit{Gizmo} Lab: Star Spectra (page 1 only)


\framebox{\textbf{Link}} \textit{Science History Institute}: ``The High-Flying, Death-Defying Discovery of Helium'' (\href{https://www.sciencehistory.org/distillations/the-high-flying-death-defying-discovery-of-helium}{click here}).

\begin{enumerate}
\setlength\itemsep{0.1ex}
    \item ``In the lab,'' at what wavelength does the sodium doublet appear?
    \item At what wavelength did Janssen observe the mysterious yellow line?
    \item Find the exact wavelengths of the sodium doublet lines. Google ``NIST strong sodium lines'' and click the first link. Note: wavelengths are in units of \SI{}{\angstrom}
    \item Find the exact wavelength of the helium yellow line. Google ``NIST strong helium lines'' and click the first link.
    \item Calculate the difference (subtract) in wavelength between the strong helium yellow line and the closet sodium doublet. 
\end{enumerate}

% \gls{blackbody}

% \gls{Wien's law}

% \gls{dispersion}

% \gls{spectrometer}

% \gls{continuous spectrum}

% \gls{absorption spectrum}

% \gls{emission spectrum}

\clearpage

\printnoidxglossaries

\end{document}