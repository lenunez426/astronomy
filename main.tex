\documentclass{article}
\usepackage{preamble}

\title{Astronomy}
\author{Luis Nunez}
\date{April 2022}

\begin{document}
\framebox{\textbf{Link}} \textit{YouTube}: ``Kregenow Astronomy Lecture \# 1, Spring 2020'' by Julia Kregenow (\href{https://youtu.be/uMQ0-Kr1ySI}{click here}). See minute \href{https://youtu.be/uMQ0-Kr1ySI?t=2872}{47:00} -- 57:00 for research results on how people learn. Components of a good class: worksheets; voting cards (ABCD) and questions; students talking to each other; brainstorming; answering out loud; asking questions; drawing names from a hat. Big astronomy ideas: science; light; lookback time; gravity; we are star stuff; the universe is dynamic; space is big and mostly empty

\section{Science and the Universe: A Brief Tour}
light-year, astronomical unit (AU)
\section{Observing the Sky: The Birth of Astronomy}
Parallax, zodiac, magnitude
\section{Orbits and Gravity}
\section{Earth, Moon, and Sky}
Right ascension (RA), declination (DEC)


\section{Radiation and Spectra}
\section{Astronomical Instruments}
\section{Other Words: An Introduction to Other Worlds}
\section{}
\section{}
\section{}
\section{}
\section{}
\section{}
\section{}
\section{}
\section{}
\section{Analyzing Starlight}
Spectral class (O, B, A, F, G, K, M), apparent magnitude
\section{}
\section{}
\section{Between the Stars: Gas and Dust in Space}
Messier catalogue, New General Catalogue (NGC),
\section{}
\section{}
\section{The Death of Stars}
Supernova

planet, moon, atmosphere, exoplanet

constellation

azimuthal, equatorial coordinates, cardinal points

satellite

meteor shower

Milky Way

solar system

Moon






\end{document}
